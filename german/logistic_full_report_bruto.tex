\begin{table}
\centering
\caption{logistic_full_report_bruto.tex}
\label{logistic_full_report_bruto.tex}
\begin{tabular}{lrrrrrrrl}
\toprule
{}              &  Brier  loss &  Log loss &  Accuracy  &  Precision  &   Recall  &       F1  &  Roc auc  & Conjunto de dados \\
\midrule
Média dos Folds &      0.16410 &  0.498400 &    0.76300 &     0.63910 &  0.473400 &  0.542200 &  0.680300 &    Conjunto bruto \\
Desv. Padrão    &      0.01796 &  0.042393 &    0.03802 &     0.08708 &  0.093961 &  0.091345 &  0.052675 &    Conjunto bruto \\
\bottomrule
\end{tabular}
\end{table}
